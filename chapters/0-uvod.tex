\begin{introduction}
V dnešní době softwarového vývoje je podstatné kontinuálně, tedy co nejčastěji, testovat vyvíjený produkt. Hlavní motivací testování je snaha neustále zvyšovat kvalitu kódu a tedy daného produktu. Čím dříve je totiž nalezen daný problém v softwaru, tím rychleji ho lze lokalizovat a opravit. Toto dřívější objevení poté může vést i k menšímu objemu kódu, ve kterém je potřeba zdroj chyby hledat. Zároveň neobjevení chyby v průběhu vývoje může mít negativní dopady na firmu, která daný produkt vytváří - od poklesu reputace až po finanční ztrátu.

Tento negativní dopad může být ještě markantnější v kontextu průmyslových zařízení. Zákazník od těchto zařízení očekává skoro nepřetržitý běh a vysokou stabilitu, protože každá minuta, kdy zařízení neběží tak, jak má, přináší jeho majiteli finanční ztrátu. V horším případě může chyba způsobená špatně fungujícím zařízením zapříčinit poškození vybavení, nebo dokonce zranění obsluhy výroby. 

Testování ovšem bývá často repetitivní činnost, proto se k testování v dnešní době hojně využívá automatizace. Díky ní jsme schopni častěji testovat, minimalizuje se faktor lidské chyby a jsme schopni lépe využít lidské zdroje, které by jinak musely manuálně testovat. Firmy proto často investují nemalé peníze do automatizace testů a jako každý ziskový subjekt se snaží svoje zisky zvyšovat snižováním vlastních nákladů. 

Jak tedy může firma snížit svoje náklady na testování? Jednou z možností je snížení počtu zúčastněných fyzických zařízení. Každé fyzické zařízení musí být zakoupeno a musí být spravováno. Zařízení jsou následně sestavována v různých topologiích. Je tedy potřeba mít dostatek zařízení, abychom pokryli všechny potřebné konfigurace. Tyto sestavy zařízení následně musí být někým spravovány a každá změna v topologii může zabrat nemálo času a úsilí.

Možností, jak tento problém vyřešit, je virtualizace testovacích zařízení a topologie zapojení těchto zařízení. Díky virtualizaci jsme schopni simulovat jakékoliv zařízení v různých počtech. Cenou přidání dalšího zařízení jsou následně pouze nutné výpočetní zdroje na zařízení, na kterém testování běží. Dostatečné výkonné zařízení může samozřejmě stát nemalé finance, ale takovéto zařízení je univerzálnější a znovu použitelné pro všechny produkty, které podporují virtualizaci.

Tato práce se věnuje návrhu a implementaci rozšíření stávající testovací knihovny, která bude podporovat virtualizaci zařízení. Knihovna by zároveň měla podporovat propojení zařízení v několika různých topologiích. Knihovna je vytvářena pro společnost Siemens s.r.o., která ji následně využije při vývoji svých produktů. 

Práce začíná kapitolou \ref{chap:cil}, ve které stanovuje cíle práce. V kapitole \ref{chap:teorie} je přiblížen teoretický kontext této práce. Následně v kapitole \ref{chap:anal} práce analyzuje stav stávající testovací knihovny a analyzuje možnosti rozšíření knihovny. Kapitola \ref{chap:design} se poté věnuje návrhu změn a rozšíření v testovací knihovně, jejichž implementace je následně popsána v kapitole \ref{chap:implementation}. V kapitole \ref{chap:demonstration} je následně ukázáno použití testovací knihovny. V neposlední řadě se kapitola \ref{chap:review} věnuje zhodnocení nové testovací knihovny. 
 
\end{introduction}