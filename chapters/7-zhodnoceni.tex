\chapter{Zhodnocení testovací knihovny}\label{chap:review}

Vytvořená testovací knihovna splňuje požadavky a cíle, které na ní byly kladeny a které byly stanoveny v kapitole \ref{chap:cil}. Vyvstává zde ovšem otázka jak efektivní je toto řešení. V následujících sekcích je tedy vytvořené řešení zhodnocena na základě několika kritérií.

\section{Kategorie hodnocení}

Vytvořená testovací knihovna bude zhodnocena na základě těchto kategorií:

\begin{itemize}
    \item Flexibilita knihovny
    \item Rychlost a efektivnost vytváření virtualizovaného prostředí
    \item Výkon virtualizované sítě
\end{itemize}


\section{Flexibilita knihovny}
Nová testovací knihovna lze využít pro jakékoliv testy, které vyžadují propojení zařízení, neboli kontejnerů, mezi sebou. Díky její implementaci je možné testy implementovat s libovolným počtem zařízení v různých topologiích.

Knihovna v aktuální podobě podporuje zařízení, které vychází z operačního systému Ubuntu. Tato podpora může být širší, protože závislost na systému Ubuntu je pouze na ty nástroje, které umožňují nastavení sítě na zařízení. Seznam kompatibilních systému ovšem zkoumám nebyl. 

Je zřejmé, že toto snižuje flexibilitu knihovny. Zároveň ovšem není náročné přidat další implementace pro jiné operační systémy. 

\section{Rychlost a efektivnost vytváření virtualizovaného prostředí}




\section{Výkon virtualizované sítě}

