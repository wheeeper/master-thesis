Docker také podporuje vlastní implementace síťových ovladačů. Ty následně mohou být nahrány do stejného repositáře jako Docker obrazy. Bohužel jelikož tyto síťové ovladače jsou zahrnuty do širší kategorie \uv{rozšíření}, tak jejich vyhledávání je náročné a často neobsahují skoro žádný popis. \cite{docker_networking_overview}\cite{docker_brige_overview}

Díky tak širokému využití Docker je jeho podpora rozšířena i pro jazyk \csharp~díky knihovnám, které byly vytvořeny komunitou. Ty umožňují přímo sestavovat a nasazovat kontejnery a spravovat virtuální síť. To velice ulehčuje jeho možné použití v testovací knihovně. 

Pro naše použití ale začínají být zajímavé až síťové typy \textit{bridge}, \textit{IPvlan} a \textit{MACvlan}. \textit{Bridge} je výchozím síťovým typem, který Docker používá v momentě kdy dva kontejnery potřebují mezi sebou komunikovat na stejném zařízení. \textit{IPvlan} oproti tomu umožňuje uživatelům plnou kontrolu nad IPv4 a IPv6 adresováním a nad druhou a třetí úrovní ISO/OSI modelu. V neposlední řadě \textit{MACvlan} umožňuje kontejnerům přiřadit různé MAC adresy, které tak mohou být odlišné od hostujícího zařízení. To umožňuje kontejnerům vypadat na síti jako fyzické zařízení.

Síťový typ \textit{host} umožňuje odstranit izolaci mezi kontejnery a plně využít síť zprostředkovanou hostem, tedy Docker serverem. Oproti tomu síťový typ \textit{overlay} slouží k propojení různých Docker daemonů a umožňuje komunikace mezi nimi. 


%podman

Podman stejně jako Docker podporuje síťové typy \textit{bridge} a \textit{MACvlan}. Ovšem k jejich použití již potřebuje práva správce počítače. K vyřešení tohoto problému Podman používá síťový typ \textit{slirp4netns}. Ta vytvoří speciální TAP zařízení, fungující na linkové vrstvě ISO/OSI modelu a umožňuje komunikaci mezi kontejnery, které jsou ovšem takto fungují izolovaně od sebe. \cite{podman_network}