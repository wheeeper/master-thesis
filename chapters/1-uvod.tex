\begin{introduction}
V dnešní době softwarového vývoje je podstatné kontinuálně, tedy co nejčastěji, testovat vyvíjený produkt. Hlavní motivací testování je snaha neustále zvyšovat kvalitu kódu a tedy daného produktu. Čím dříve je nalezen daný problém v softwaru, tím rychleji ho lze lokalizovat a opravit. Dřívější objevení může vést i k menšímu objemu kódu, ve kterém je potřeba zdroj chyby hledat. Zároveň neobjevení chyby v průběhu vývoje může mít negativní dopady na firmu, která daný produkt vytváří - od poklesu reputace až po finanční ztrátu.

Testování ovšem bývá často repetitivní činnost, proto se k testování v dnešní době hojně využívá automatizace. Díky ní jsme schopni častěji testovat, minimalizuje se faktor lidské chyby a jsme schopni lépe využít lidské zdroje, které by jinak musely manuálně testovat. Firmy proto často investují nemalé peníze do automatizace testů a jako každý ziskový subjekt, se snaží svoje zisky zvyšovat snižováním vlastních nákladů. 

Jak tedy může firma snížit svoje náklady na testování? Jedna z možností je snaha snížit počet zúčastněných fyzických zařízení. Každé fyzické zařízení musí být zakoupeno a musí být spravováno. Zařízení následně jsou sestavována v různých topologiích. Je tedy potřeba mít dostatek zařízení, abychom pokryli všechny potřebné konfigurace. Tyto sestavy zařízení následně musí být někým spravovány a každá změna v topologii může zabrat nemalý čas a úsilí.

Jednou z možností, jak tento problém vyřešit, je virtualizace testovacích zařízení a topologie zapojení těchto zařízení. Díky virtualizaci jsme schopni simulovat různá zařízení v různých počtech. Cena přidání nového zařízení jsou pouze nutné výpočetní zdroje na zařízení, na kterém testování běží. Dostatečné výkonné zařízení může samozřejmě stát nemalé finance, ale toto zařízení je univerzálnější a znovu použitelné pro všechny produkty, které podporují virtualizaci.

\customtodo{Doplnit obsah kapitol}
\end{introduction}