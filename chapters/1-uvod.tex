\begin{introduction}
V dnešní době softwarového vývoje je podstatné kontinuálně, tedy co nejčastěji, testovat vyvíjený produkt. Na otázku proč co nejčastěji je jednoduchá odpověď. Čím dřív nalezneme daný problém v softwaru, tím rychleji ho můžeme lokalizovat a opravit. Dřívější objevení zároveň většinou znamená menší objem kódu, ve kterém musíme zdroj chyby hledat.

K splnění těchto požadavků je hojně používaná automatizace testů. Díky automatizaci jsme schopni testovat mnohem častěji než manuálně a zároveň minimalizujeme faktor lidské chyby při manuálním testování. Firmy proto často investují nemalé peníze do automatizace testů a jako každý ziskový subjekt, se snaží svoje zisky zvyšovat snižováním vlastních nákladů. 

Jak tedy může firma snížit svoje náklady na testování? Jedna z možností je snaha snížit počet zúčastněných fyzických zařízení. Každé fyzické zařízení musí být zakoupeno a musí být spravováno. Zároveň každé zařízení může v industriálním prostředí existovat v různých konfiguracích. To též musí být otestováno, proto pokud chceme automaticky testovat různé konfigurace, tak poté musí existovat různé sestavy v různých konfiguracích. 

A jak tento problém vyřešit? Jednou možností, je virtualizace testovacích zařízení a topologie zapojení těchto zařízení. Díky virtualizaci jsme schopni simulovat různá zařízení v různých počtech. Cena přidání nového zařízení jsou pouze nutné výpočetní zdroje na zařízení, na kterém testování běží. Dostatečné výkonné zařízení může samozřejmě stát nemalé finance, ale toto zařízení je univerzálnější a znovu použitelné pro všechny produkty, které podporují virtualizaci.

\customtodo{Doplnit obsah kapitol}
\end{introduction}