\begin{conclusion}
Cílem této práce bylo analyzovat, navrhnout a implementovat rozšíření testovací knihovny o virtualizační prvky, které ji měli umožnit vytvářet zařízení ve virtualizovaném prostředí a propojovat je ve třech různých topologiích. Práce také měla demonstrovat způsob použití nové knihovny a zhodnotit její vlastnosti a možnosti reálného použití.

Cíle této práce byly splněny. Práce v kapitole \ref{chap:teorie} definuje jednotlivé pojmy a přibližuje kontext této práce. V kapitole \ref{chap:anal} práce zhodnocuje stav předchozí knihovny a analyzuje možnosti jejího rozšíření. Následně v kapitole \ref{chap:design} práce navrhuje změny, dle výstupu z analýzy, které jsou následně implementovány v kapitole \ref{chap:implementation}. V kapitole \ref{chap:demonstration} práce ukazuje možné použití knihovny s pomocí průmyslového protokolu ModbusTCP. V neposlední řadě práce zhodnocuje vlastnosti nové knihovny v kapitole \ref{chap:review}.

Vytvořené řešení splňuje požadavky, které na něj byly kladeny, avšak jak již bylo zmíněno v kapitole \ref{chap:review}, stále zde existuje prostor pro zlepšení. Knihovnu je ovšem ve svém aktuálním stavu možné využít při reálném vývoji. 
\end{conclusion}